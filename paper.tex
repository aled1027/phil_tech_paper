\documentclass[11pt]{article}

\usepackage{csquotes}
\usepackage{color}

% Alex L's macros 
% taken and co-opted from a variety of sources

% require \usepackage{color}

% generic commands
%\newcommand{\NN}{\mathbb{N}}
%\newcommand{\RR}{\mathbb{R}}
%\newcommand{\compindist}{\approx_C}

% definition counter
%\newcounter{defcounter}
%\setcounter{defcounter}{0}
%\newenvironment{definition}{\medskip\noindent\refstepcounter{defcounter}{\bf Definition \thedefcounter}\hspace*{2pt}}{\hspace*{\fill}\nopagebreak[4]$\diamondsuit$\medskip}  

% block quote
%\newenvironment{blockquote}{%
%  \par%
%  \medskip
%  \leftskip=4em\rightskip=2em%
%  \noindent\ignorespaces}{%
%  \par\medskip}
%
%% Author Macros
%% colors: red, magenta, blue, orange
\newcounter{al}
\newcommand{\al}[1]{\textcolor{blue}{\{AL-\arabic{al}: #1\}}\addtocounter{al}{1}}
%\newcounter{cat}
%\newcommand{\cat}[1]{\textcolor{magenta}{\{CAT-\arabic{cat}: #1\}}\addtocounter{cat}{1}}
%
%\newcommand{\ignore}[1]{}
%
%% From CompGC paper
%%\renewcommand{\sim}{S}
%%\newcommand{\Input}{\ensuremath{\textsf{Input}}\xspace}
%%\newcommand{\Output}{\ensuremath{\textsf{Output}}\xspace}
%%\newcommand{\Inputs}{\ensuremath{\textsf{Inputs}}\xspace}
%%\newcommand{\Outputs}{\ensuremath{\textsf{Outputs}}\xspace}
%%\newcommand{\Components}{\ensuremath{\textsf{Components}}\xspace}
%
%
%% my commands for JustGarble Notation
%\newcommand{\Gates}{\text{Gates}}
%\newcommand{\InputWires}{\text{InputWires}}
%\newcommand{\OutputWires}{\text{OutputWires}}
%\newcommand{\Wires}{\text{Wires}}
%\newcommand{\A}{\mathcal{A}}
%\newcommand{\Enc}{\textsf{Enc}} % or mathsf
%\newcommand{\Dec}{\textsf{Dec}}
%\newcommand{\Gen}{\textsf{Gen}}
%\newcommand{\EncInv}{\Enc^{-1}}
%\newcommand{\EncDKC}{\textsf{EncDKC}}
%\newcommand{\DecDKC}{\textsf{DecDKC}}
%\newcommand{\EncDKCInv}{\EncDKC^{-1}}
%\newcommand{\compIndist}{\approx_D}
%\newcommand{\outputrv}{{\sf output}}
%\newcommand{\viewrv}{{\sf view}}
%
%% Garbled circuits notations
%\newcommand{\CompGC}{\textsf{CompGC}\xspace}
%\newcommand{\JustGarble}{\textsf{JustGarble}\xspace}
%\newcommand{\Naive}{\textsf{Naive}\xspace}
%\newcommand{\scmc}{SCMC\xspace} % Single Communication Multiple Connnection


\title{Creative Destruction}
\author{Alex Ledger}
%\date{}
\pagestyle{headings}

\begin{document}
\maketitle

\section{Introduction and Old Thoughts on Creative Destruction}
Creative destruction is economic idea that refers to how new products, innovations, firms and regimes destroy old products, firms and regimes.
The idea has its roots in Marx, who questions why capitalism must destroy itself in order to exist.
Many have expanded on Marx ideas.
Most notably is Schumpeter in 1942 with his book \textit{Capitalism, Socialism and Democracy}.

Briefly, Schumpeter believes that a capitalistic economic system cannot be stationary; rather, it must be continuously in flux. 
The fundamental driver of the continuous change is new consumer good, new production methods, new markets, etc. 
The necessary new-ness of things creates ``the process of industrial mutation that incessantly revolutionizes the economic structure from within, incessantly destroying the old one, incessantly creating a new one'' \cite{wikipedia quote on schumpeter}.
Schumpter, like Marx, was skeptical of whether the creative destructive nature of capitalism was sustainable. 
Eventually, there could be too much destruction such that the system can no longer persist.

In our readings for philosophy of technology, Jaffe and Trachtenburg introduced us to concept of creative destruction.
Their aim is to formalize the idea of creative destruction and complete empirical investigations:
\begin{displayquote}
    Our aim in this paper is to create a framework for incorporating the microeconomics of creative destruction and knowledge spillovers into a model of growth, and to do so in such a way that we can begin to measure them and untangle the forces that determine their intensity and impact on growth (90)
\end{displayquote}

They analyze creative destruction empirically by using the U.S. patents database and data on market value of industries (91).
Their model is complicated, but I believe their process is straightforward enough: they correlation between patents issued from a particular sector and the market value of that sector \al{get page number on this}. 
\begin{displayquote}
    “That is, in an average sector at an average year a firm that does not invent sees its value relative to that of the industry erode by about 4\%.” (91)
\end{displayquote}


\newpage
\section{New Thoughts on Creative Destruction}
This section will discuss my thoughts on creative destruction.

\end{document}















