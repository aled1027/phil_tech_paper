\documentclass[11pt]{article}

\usepackage{csquotes}
\usepackage{color}
\usepackage{amsfonts}


% Alex L's macros 
% taken and co-opted from a variety of sources

% require \usepackage{color}

% generic commands
\newcommand{\NN}{\mathbb{N}}
\newcommand{\RR}{\mathbb{R}}
%\newcommand{\compindist}{\approx_C}


% block quote
%\newenvironment{blockquote}{%
%  \par%
%  \medskip
%  \leftskip=4em\rightskip=2em%
%  \noindent\ignorespaces}{%
%  \par\medskip}
%
%% Author Macros
%% colors: red, magenta, blue, orange
\newcounter{al}
\newcommand{\al}[1]{\textcolor{blue}{\{AL-\arabic{al}: #1\}}\addtocounter{al}{1}}
\newcommand{\T}{\mathcal{T}}

\newtheorem{theorem}{Theorem}[section]
\newtheorem{lemma}[theorem]{Lemma}
\newtheorem{proposition}[theorem]{Proposition}
\newtheorem{corollary}[theorem]{Corollary}

\newenvironment{proof}[1][Proof]{\begin{trivlist}
\item[\hskip \labelsep {\bfseries #1}]}{\end{trivlist}}
\newenvironment{definition}[1][Definition]{\begin{trivlist}
\item[\hskip \labelsep {\bfseries #1}]}{\end{trivlist}}
\newenvironment{example}[1][Example]{\begin{trivlist}
\item[\hskip \labelsep {\bfseries #1}]}{\end{trivlist}}
\newenvironment{remark}[1][Remark]{\begin{trivlist}
\item[\hskip \labelsep {\bfseries #1}]}{\end{trivlist}}

\newcommand{\qed}{\nobreak \ifvmode \relax \else
      \ifdim\lastskip<1.5em \hskip-\lastskip
      \hskip1.5em plus0em minus0.5em \fi \nobreak
      \vrule height0.75em width0.5em depth0.25em\fi}
%\newcounter{cat}
%\newcommand{\cat}[1]{\textcolor{magenta}{\{CAT-\arabic{cat}: #1\}}\addtocounter{cat}{1}}
%
%\newcommand{\ignore}[1]{}
%
%% From CompGC paper
%%\renewcommand{\sim}{S}
%%\newcommand{\Input}{\ensuremath{\textsf{Input}}\xspace}
%%\newcommand{\Output}{\ensuremath{\textsf{Output}}\xspace}
%%\newcommand{\Inputs}{\ensuremath{\textsf{Inputs}}\xspace}
%%\newcommand{\Outputs}{\ensuremath{\textsf{Outputs}}\xspace}
%%\newcommand{\Components}{\ensuremath{\textsf{Components}}\xspace}
%
%
%% my commands for JustGarble Notation
%\newcommand{\Gates}{\text{Gates}}
%\newcommand{\InputWires}{\text{InputWires}}
%\newcommand{\OutputWires}{\text{OutputWires}}
%\newcommand{\Wires}{\text{Wires}}
%\newcommand{\A}{\mathcal{A}}
%\newcommand{\Enc}{\textsf{Enc}} % or mathsf
%\newcommand{\Dec}{\textsf{Dec}}
%\newcommand{\Gen}{\textsf{Gen}}
%\newcommand{\EncInv}{\Enc^{-1}}
%\newcommand{\EncDKC}{\textsf{EncDKC}}
%\newcommand{\DecDKC}{\textsf{DecDKC}}
%\newcommand{\EncDKCInv}{\EncDKC^{-1}}
%\newcommand{\compIndist}{\approx_D}
%\newcommand{\outputrv}{{\sf output}}
%\newcommand{\viewrv}{{\sf view}}
%
%% Garbled circuits notations
%\newcommand{\CompGC}{\textsf{CompGC}\xspace}
%\newcommand{\JustGarble}{\textsf{JustGarble}\xspace}
%\newcommand{\Naive}{\textsf{Naive}\xspace}
%\newcommand{\scmc}{SCMC\xspace} % Single Communication Multiple Connnection


\title{Creative Destruction}
\author{Alex Ledger}
%\date{}
\pagestyle{headings}

\begin{document}
\maketitle

\section{Old Thoughts on Creative Destruction}
\subsection{Marx and Schumpter}
This section introduces creative destruction.
Creative Destruction has its roots in economics, and as such, this section will present creative destruction in its economic form.
In effect, this section lacks philosophical rigor.

Creative destruction is an economic idea that refers to how new products, firms and innovations destroy old products, firms and innovations.
Neoclassical economic theory, the most popular economic theory, is built on top of the idea of equilibrium.
Most simply, an economy has supply of things and demands of the same things. 
Supply and demand interact with each other, creating tension between economic processes.
The economy remains in flux until supply and demand reach a happy equilibrium.
At a happy equilibrium, supply and demand remain more or less constant until something disrupts the system.

The supply and demand theory has had strong explanatory power over the years, but it doesn't expalin one crucial observation: economies grow.
In a neoclassical economic framework, an economy reaches equilibrium and stabilizes.
Once an economy stabilizes, the only for it to grow is via exogenous inputs, for example, imperialism, increased supply of raw materials.

However, we observe that economies grow faster than the net exogenous inputs.
What could explain this?
Better worded: what explains endogenous growth in an economy?
After a close reading of Marx, Schumpeter came to the idea that innovation and entrepreneurship are the drivers of endogenous growth.
Innovation and entrepreneurship introduce new value to the economy that did not previously exist, contributing to an overall increase in social wealth.
However, Schumpeter also observed that new technologies (that is, the result of innovation)  and new firms often displace old technologies and old technologies.
The new technologies and new firms are better - specifically they have a lower margincal cost or are more desirable widgets - and the old technologies lose economic value and the old firms lose profits.

This the idea of creative destruction.
The old displaces new; there is not room for both of them.
Marx and Schumpeter had interesting concern: what if there is too much destruction by new technology such that society destroys itself.
The capitalist system is a cycle of creation and destruction, and it seems possible, on a theoretical level, that the destruction may overwhelm the construction, spelling disaster for society.
While on a social level this is hard to imagine, it is somewhat easier to fathom on technological level.

This paper will explore the idea of creative destruction and its philosophical implications on technology. 
To whet the appetite, a few questions that will be addressed are:
\begin{itemize}
    \item New technology technically does not ``destroy'' old technology; rather, it makes the old technology lose economic value.
    \item But, what effect does new technology have on old technology? Does it make it obsolete? Does the ``idea'' of the old technology change? Does the ``meaning'' of technology, which is contextual and socially constructed, changed due to a change in society or due to a change in technology? That question implies that technosphere, as one might call it, is distinct from society, which social constructs persuasively argue is not necessarily the case.
    \item Is technology still cumulative? Or in what nuanced sense is it cumulative?
    \item How does this idea fit into evolution?
    \item What role does this play in selection?
    \item Any effect on combination? gut says no, but not sure
\end{itemize}

\subsection{Caballero and Jaffe}
In our readings for philosophy of technology, Caballero and Jaffe introduced us to concept of creative destruction.
Their aim is to formalize the idea of creative destruction and knowledge spillovers and complete empirical investigations:
\begin{displayquote}
    Our aim in this paper is to create a framework for incorporating the microeconomics of creative destruction and knowledge spillovers into a model of growth, and to do so in such a way that we can begin to measure them and untangle the forces that determine their intensity and impact on growth (90)
\end{displayquote}

Caballero and Jaffe view the economy in a schumpeterian way: ``Schumpeter recognized that innovation was the engine of growth, and that innovation is endogenously generated by competing profit-seeking firms'' (90). 
More formally, they say the economy consists of ``a continuum of monopolistically competitive good indexed by their quality $q \in (-\infty, N_t]$''.
In the economy, a firm that operates with constant marginal cost, i.e. a firm that does not innovating, will see their profits decline. 
If new good are more substitutable for old goods, then the firm will see their profits decline more quickly.

Beyond innovation, they are interested in knowledge spillovers, believing that increases in public knowledge contribute to economy wide returns.
They distinguish between knowledge and technological obsolescence and creative destruction:
\begin{displayquote}
``Old knowledge eventually is made obsolete by the emergence of newer, superior knowledge. We call this phenomenon `knowledge' or `technological' obsolescence, and distinguish it from the obsolescence in value represented by creative destruction.
That is, new ideas have two distinct effects on the current stock of ideas. 
They make the products represented by those ideas less valuable (creative destruction or value obsolescence), and they make the knowledge represented by those ideas less relevant in the production of new knowledge (knowledge or technological obsolescence). 
The strength of knowledge spillovers, and hence the growth of the economy, will depend on the parameters of the processes of knowledge diffusion and knowledge obsolescence.'' (92)
\end{displayquote}
Creative destruction is specifically the effect of new innovation on the economic ``value'' of products.

From a philosophy perspective, it seems overly limiting to only consider creative destruction as the impact that new creations have on the economic value of old technology.
As a contrived example, consider Intel who sells CPUs. 
If Intel creates 

\section{Creative Destruction in the Theory of Technological Evolution}
The aim of this section to situate creative destruction in the theory of technological evolution.
Many authors have different ideas of technological evolution; however, the important components for this discussion is that there is some selection process. 
The theory of technological evolution imagines that many technologies exist, or potential exist inside of an inventor's head, and that some social force selects which technologies are useful to humans and which aren't.
This section will treat the aim of selection as ``use to humans'', meaning that if a technology is selected, then we say that technology was considered ``useful to humans'' in that particular social context. 
Brian Arthur was the most recent author we read to hold this view of selection.

Creative Destruction, as described by Jaffe, is the idea that sometimes when a new technology comes into existence, the new technology decreases the economic value of an old technology. 
\al{motivate usage of use-value over economic value}

Let us take a step back for a second.
Imagine that at some point in time, there exists technologies in the world - call this set of existing technologies $\T$. 
Each $t \in \T$ has an use-value at this point in time. 
Then a new technology $s$ is created.
Upon the creation of $u$, the use-value of each technology $t$ in $\T$ is subject to change.
The use-value of $t$ may increase, indicating that $u$ complements $t$: $u$ and $t$ function well together and make each other more valuable.
The use-value of $t$ may stay the same, indicating the $u$ and $t$ are not related.
Or the use-value of $t$ may decrease, indicating that $u$ substitutes $t$: $u$ may perform that same job as $t$, and hence decrease the use-value of $t$, or something along these lines ($u$ may be complementary to a replacement of $t$ - the causal relationship can be complicated). 

Creative destruction is concerned with the third case: where a new technology $u$ reduces the use-value of an old technology $t$. 


%\begin{definition}
%Let $\T$ be the set of all existing technologies. 
%Let $V: \T \to \RR_{\geq 0}$ be the value mapping, where $V(t)$ is the economic value of a technology.
%Let $t$ be a new technology.
%Then upon the existence of $t$, there exists a new value mapping $V'$, mapping technologies $\T \cup \{t\}$ to their new economic value.
%The addition of $t$ to $\T$ affects the economic value of all technologies in $\T$.
%\end{definition}



\subsection{What is Creative Destruction and how does it fit the theory?}
\subsection{Implications on the Theory of Technology}
\section{Philosophical Consdiderations of Creative Destruction}

\end{document}















