\documentclass[11pt]{article}

\usepackage{csquotes}
\usepackage{color}
\usepackage{amsfonts}


% Alex L's macros 
% taken and co-opted from a variety of sources

% require \usepackage{color}

% generic commands
%\newcommand{\NN}{\mathbb{N}}
%\newcommand{\RR}{\mathbb{R}}
%\newcommand{\compindist}{\approx_C}

% definition counter
%\newcounter{defcounter}
%\setcounter{defcounter}{0}
%\newenvironment{definition}{\medskip\noindent\refstepcounter{defcounter}{\bf Definition \thedefcounter}\hspace*{2pt}}{\hspace*{\fill}\nopagebreak[4]$\diamondsuit$\medskip}  

% block quote
%\newenvironment{blockquote}{%
%  \par%
%  \medskip
%  \leftskip=4em\rightskip=2em%
%  \noindent\ignorespaces}{%
%  \par\medskip}
%
%% Author Macros
%% colors: red, magenta, blue, orange
\newcounter{al}
\newcommand{\al}[1]{\textcolor{blue}{\{AL-\arabic{al}: #1\}}\addtocounter{al}{1}}
%\newcounter{cat}
%\newcommand{\cat}[1]{\textcolor{magenta}{\{CAT-\arabic{cat}: #1\}}\addtocounter{cat}{1}}
%
%\newcommand{\ignore}[1]{}
%
%% From CompGC paper
%%\renewcommand{\sim}{S}
%%\newcommand{\Input}{\ensuremath{\textsf{Input}}\xspace}
%%\newcommand{\Output}{\ensuremath{\textsf{Output}}\xspace}
%%\newcommand{\Inputs}{\ensuremath{\textsf{Inputs}}\xspace}
%%\newcommand{\Outputs}{\ensuremath{\textsf{Outputs}}\xspace}
%%\newcommand{\Components}{\ensuremath{\textsf{Components}}\xspace}
%
%
%% my commands for JustGarble Notation
%\newcommand{\Gates}{\text{Gates}}
%\newcommand{\InputWires}{\text{InputWires}}
%\newcommand{\OutputWires}{\text{OutputWires}}
%\newcommand{\Wires}{\text{Wires}}
%\newcommand{\A}{\mathcal{A}}
%\newcommand{\Enc}{\textsf{Enc}} % or mathsf
%\newcommand{\Dec}{\textsf{Dec}}
%\newcommand{\Gen}{\textsf{Gen}}
%\newcommand{\EncInv}{\Enc^{-1}}
%\newcommand{\EncDKC}{\textsf{EncDKC}}
%\newcommand{\DecDKC}{\textsf{DecDKC}}
%\newcommand{\EncDKCInv}{\EncDKC^{-1}}
%\newcommand{\compIndist}{\approx_D}
%\newcommand{\outputrv}{{\sf output}}
%\newcommand{\viewrv}{{\sf view}}
%
%% Garbled circuits notations
%\newcommand{\CompGC}{\textsf{CompGC}\xspace}
%\newcommand{\JustGarble}{\textsf{JustGarble}\xspace}
%\newcommand{\Naive}{\textsf{Naive}\xspace}
%\newcommand{\scmc}{SCMC\xspace} % Single Communication Multiple Connnection


\title{Creative Destruction}
\author{Alex Ledger}
%\date{}
\pagestyle{headings}

\begin{document}
\maketitle

\section{Old Thoughts on Creative Destruction}
\subsection{Marx and Schumpter}
This section introduces creative destruction.
Creative Destruction has its roots in economics, and as such, this section will present creative destruction in its economic form.
In effect, this section lacks philosophical rigor.

Creative destruction is an economic idea that refers to how new products, firms and innovations destroy old products, firms and innovations.
Neoclassical economic theory, the most popular economic theory, is built on top of the idea of equilibrium.
Most simply, an economy has supply of things and demands of the same things. 
Supply and demand interact with each other, creating tension between economic processes.
The economy remains in flux until supply and demand reach a happy equilibrium.
At a happy equilibrium, supply and demand remain more or less constant until something disrupts the system.

The supply and demand theory doesn't explain one crucial observation: economies grow.
In a neoclassical economic framework, an economy reaches equilibrium and stabilizes.
Once an economy stabilizes, the only for it to grow is via exogenous inputs, for example, imperialism, increased supply of raw materials.

However, we observe that economies grow faster than the net exogenous inputs.
What could explain this?
Better worded: what explains endogenous growth in an economy?
After a close reading of Marx, Schumpeter came to the idea that innovation and entrepreneurship are the drivers of endogenous growth.
Innovation and entrepreneurship introduce new value to the economy that did not previously exist, contributing to an overall increase in social wealth.
However, Schumpeter also observed that new technologies (that is, the result of innovation)  and new firms often displace old technologies and old technologies.
The new technologies and new firms are better - specifically they have a lower margincal cost or are more desirable widgets - and the old technologies lose economic value and the old firms lose profits.

This the idea of creative destruction.
The old displaces new; the better, more efficient verison makes the old, costly version unviable.
For example, the smartphone, sophisticated little computer with the power the text and call, destroyed the economic viability of the dumb-phone.
Marx and Schumpeter were concerned about the possible effects of creative destruction: what if there is too much destruction?
The capitalist system is a cycle of creation and destruction, and it seems possible, on a theoretical level, that the destruction may overwhelm the construction, spelling disaster for society.

\subsection{Caballero and Jaffe}
In our readings for philosophy of technology, Caballero and Jaffe introduced us to concept of creative destruction.
Their aim is to formalize the idea of creative destruction and knowledge spillovers and complete empirical investigations:
\begin{displayquote}
    Our aim in this paper is to create a framework for incorporating the microeconomics of creative destruction and knowledge spillovers into a model of growth, and to do so in such a way that we can begin to measure them and untangle the forces that determine their intensity and impact on growth (90)
\end{displayquote}

Caballero and Jaffe view the economy in a schumpeterian way: ``Schumpeter recognized that innovation was the engine of growth, and that innovation is endogenously generated by competing profit-seeking firms'' (90). 
More formally, they say the economy consists of ``a continuum of monopolistically competitive good indexed by their quality $q \in (-\infty, N_t]$''.
In the economy, a firm that operates with constant marginal cost, i.e. a firm that does not innovating, will see their profits decline. 
If new good are more substitutable for old goods, then the firm will see their profits decline more quickly.

Beyond innovation, they are interested in knowledge spillovers, believing that increases in public knowledge contribute to economy wide returns.
They distinguish between knowledge and technological obsolescence and creative destruction:
\begin{displayquote}
``Old knowledge eventually is made obsolete by the emergence of newer, superior knowledge. We call this phenomenon `knowledge' or `technological' obsolescence, and distinguish it from the obsolescence in value represented by creative destruction.
That is, new ideas have two distinct effects on the current stock of ideas. 
They make the products represented by those ideas less valuable (creative destruction or value obsolescence), and they make the knowledge represented by those ideas less relevant in the production of new knowledge (knowledge or technological obsolescence). 
The strength of knowledge spillovers, and hence the growth of the economy, will depend on the parameters of the processes of knowledge diffusion and knowledge obsolescence.'' (92)
\end{displayquote}
Creative destruction is specifically the effect of new innovation on the economic ``value'' of products.
Caballero and Jaffe measures the effect on innovation on industries by using market value and patent data on 567 large U.S. firms.
They found that:
\begin{displayquote}
    ``In an average sector at an average year a firm that does not invent sees it relative value to that of the industry erode by about 4\%.''
\end{displayquote}
That is, there is approximately 4\% destruction of pre-existing technology on average per year. 
Different industries observed varying levels of destruction.
On the high end, the pharmaceutical industry observed 25\% destruction per year, yet multiple sectors observed 0\% destruction.

We find Caballero and Jaffe's method to be interesting and the results to be useful, but the story is not yet complete.
The model that Caballero and Jaffe use to compute creative destruction\footnote{The model is not communicated in this paper} is complicated and computed over a short number of years.
However, the findings are consistent with intuition.
4\% destruction on average per year is a reasonable number, and it is to be expected that pharmaceuticals see a high level destruction relative to other industries. 
Further research could be done on the economic side of creative destruction.
First, Jaffe and Caballero could further investigate the nature of the technologies that are being destroyed, charactering the nature of the destruction (e.g. is it intra-firm created, extra-firm created, expected destruction, unexpected destruction, and so on.).
Second, it would be interesting to measure creative destruction using other methods, for example using research and development expenditures.
Third, it would be interesting to see how the industries are effected by the varying degrees of creative destruction, and the way that the firms innovate.

We greatly appreciate Caballero and Jaffe's results, and use them to motivate our research and argue about the effects of creative destruction.

\section{Creative Destruction in the Theory of Technological Evolution}
The aim of this section to situate creative destruction in the theory of technological evolution.
Many authors have different ideas of technological evolution; however, the important component for this discussion is that there is some selection process by which fit technologies remain relevant to society.
The aim of the selection process is an open question.
Economists like Jaffe believe that technologies are selected because of their potential for economic profits \cite{jaffe}.
Social constructivists believe that technologies are selected because of a complex causal web involving society, social conventions, economics, history, psychology and so on \cite{scot}.
For our treatment of technological evolution, we do not believe it be too important how selection, but rather, it is important that selection occurs. 
With this in mind, we use Brian Arthur's simple explanation of selection: a technology is useful to humans \cite{arthur}.
We say that a particular technology has a use-value corresponding to how useful the technology is to society, humans, or the economy. 

We now give a formal description of the selection process using the idea of use-value. 
At a point in time, there exist a set of technologies in society - call this set of existing technologies $\T$. 
Each technology $t \in \T$ has an use-value at this point in time. 
Suppose that a new technology $u$ is created.
Upon the creation of $u$, the use-value of each technology $t$ in $\T$ is subject to change.
The use-value of $t$ may increase, indicating that $u$ complements $t$: $u$ and $t$ function well together and make each other more valuable.
The use-value of $t$ may stay the same, indicating the $u$ and $t$ are not related.
Or the use-value of $t$ may decrease, indicating that $u$ substitutes $t$: $u$ may perform that same job as $t$, and hence decrease the use-value of $t$, or something along these lines ($u$ may be complementary to a replacement of $t$ - the causal relationship can be complicated)
\footnote{We assume that use-value is 1-dimensional value: a use-value can go up or down, but not left and right. 
This simplification appears to be fine, but in the future, it may be useful to think of useful as a vector in vector space, allowing for more abstraction relationships and changes.}.

Creative destruction is concerned with the third case: where a new technology $u$ reduces the use-value of an old technology $t$. 
Biologists identify a similar process to creative destruction occurring in the selection of species. 
The phenomenon is called \textit{competitive exclusion}. 
Competitive exclusion is the idea that if two species are competing for the same resource, then one of the species will overcome the other, forcing the other species to adapt or die \cite{wiki}. 
In recent years, observational, experimental and simulation-based evidence has surfaced to cause Biologists to question the simplicity of competitive exclusion.
We did not examine the new evidence close enough to explain here.

There is a strong a parallel between creative destruction and competitive exclusion.
To use the terms of Biology, two technologies cannot coexist and rely on the same resources for their use-value, so one must either die or adapt. 
Technologies do not die or adapt per se, rather they either become irrelevant.
And it is still an unanswered question whether a technology adapts. 
To Biologists, the subject which is adapting is the species, and there is no straightforward analog of a species in technology. 
We ignore the issue and say there is no concept of a species in technology); rather, when a technology adapts in response to competition, the technology is really giving birth a new technology, and that new technology is more fit than its parent.

The story of competitive exclusion applies well to the story of creative destruction.
Does the opposite hold true?
A technologist (us) would say that upon the creation of a new species (or upon the creation of a new adaptation of a species) the use-value of all other species are affected, and species whose use-values decline are in some sense destroyed. 
The use-value in this context is fitness - reproductive success.
The species whose fitness declines sufficiently may become extinct; otherwise, the species adapt, changing their fitness. 
The story seems of creative destruction also seems to apply well to competitive exclusion.

The one caveat in both stories is the usage of use-value and fitness. 
The definition of use-value given at the beginning of this section focused on how useful a certain technology was to society. 
This idea has no direct relation to the reproductive power of a given technology. 
A technology that is more likely to reproduce is likely correlated to its use-value (although that would need to be tested empirically), but there exist technologies whose use-value remains high but they do not reproduce. 
The best example of this is the MRI.
The MRI was invented in the 1970s, and since then it has been used widely (i.e. high/constant use-value), but few people have innovated on top of it (i.e. low reproductive success). 
It is quite possible that the MRI is an anomaly, and that in general, reproductive success is strongly correlated to use-value. 
We believe this to be true.
Use-value is highly correlated to reproductive success, and use-value is a sufficient metric of fitness.

If use-value and reproductive success are not highly correlated, then there are a number of steps one could take.
First, one could use reproductive success instead of use-value when describing selection and creative destruction. 
The theory would likely hold, but there many be detailed differences. 
Second, one could use this as support that technological evolution is not darwinian: the important factor in technological evolution is not reproduction but rather the use-value of a technology
\footnote{This is an interesting idea. Reproductive power may not be a central component of evolutionary systems.}.

\subsection{Implications on the Theory of Technology}
This section discusses the implications of creative destruction on the theory of technology.

The first implication of creative destruction is that faster invention creation results in faster destruction.
The idea is that as society innovates more, creates new inventions, then there is simultaneously a destruction of old technology. 
And since society is innovating more, presubly there is more destruction.
There may be a net-effect of more innovation, but it is important that destruction may increase in periods of rapid innovation.
This suggests that technology is subject a phenomenon like Newton's third law: the more technology there is, the greater the push-back on old technology. 

The second implication of creative destruction is that there could be a net-decrease in technology. 
The idea of creative destruction leaves open the possibility that more technologies are destroyed than are created.
That being said, as far as we are aware, a net-decrease in technology has not been observed.
Yet this is an idea that worred Marx and Schumpeter, albeit their concern was with capitalism not technology.
Marx and Schumpeter worried that capitalism would destroy itself; there would be so much sudden growth destroying the old system, and the system would collapse.

We find it imaginable that a new series of technology destroy old technology too quickly, rendering the new and old technology useless.
Consider many sudden innovations to the internet.
The internet is a highly complex, abstract, layered, decentralized web of interactions on which much technology relies.
One could imagine that we, society, decide to improve the internet, but improve it such that the old infrastructure is destroyed and the new infrastructure insufficient or incomplete.
We, society, have buried ourselves into a hole.
We, society, don't have reliable technology to use. 
We can't simply reimplement the old internet because it is too intertwined, too complicated.
And we can't use the new internet for whatever reason.
In the end, more was destroyed than created.

\section{Philosophical Considerations of Creative Destruction}
This section discusses philosophical considerations of creative destruction.
We outline three philosphical implications.

First, creative destruction makes technology more unpredictable.
The reason for this hearkens back to Brian Arthur's idea of primitives. 
Arthur reasons that new technology is the product of a new combination of tecnological primitives. 
At a given time, there exists a set of primitives. 
A new technology is created out of recombining the existing primitives, that the new technology is added to set of the primitives.
Creative destruction suggests that when a new technology is created and added to the set of primitives, pre-existing technologies may be removed from the set of primitives.
The primitives interact with each other - some in a complementary way and some in a supplementary way (like creative destruction). 
When one is trying to predict the future path of technology, the primitives available in the future may be missing primitves that are present now, making the task substantially more complex.

Second, creative destruction does introduce a feeling that technology is dangerous.
Mark and Schumpeter worried that society would create so much that it would consume and destroy itself, and this is a possibility with technology.
In section 2, we outlined an example of how the internet may be destroyed if there is too much innovation too fast.
New technology, introduced and implmented in the wrong way, could undermine the technological infrastructure that has been built up. 
With the technological infrastructure gone, it's not difficult to imagine the negative social reprecussions.

Third, we have the perhaps unfortunate situation where there is constantly new technology.
New iphones, new computers, new drugs.
With the creation of each new invention, creative destruction theory says that old technology may be destroyed.
The old technology, which was likely less expensive (for various reason), is now inaccessible: all that exists is the new, expensive technology.
The implication is that the poor, who cannot afford the new technology, are left with nothing. 
In cases, the poor can't afford the new technology, and the old technology, which they may or may not be able to afford, no longer exists.

The most pressing area where we see this problem is the pharmaceutical industry.
There are two phases for a drug on the market. 
The first phase is where the drug is patented, the firm has a monopoloy on it and prices are high.
The second phase is after the patent expires, and the drug enters the public domain. 
Unfortunatley, some drugs don't benefit from the second phase. 
\footnote{I did not look up examples and examine this in detail.}
There are cases where people don't benefit from drugs being the second phase.
It is not economically viable for firms to mass produce the drug, so poorer people can't access it. 

\section{Conclusion}
Creative Destruction, the idea that new technologies make existing technologies less relevant, fits in nicely with the technological theory of evolution.
In particular, creative destruction explains a portion of the selection process, and how new technologies affect the fitness landscape of existing technologies.
Moreover, creative destruction has a strong analog in biological evolution, competitive exclusion.
Creative destruction and competitive exclusion share many commonalities and few differences, but the similarities suggest that at least some part of technological selection is quite similar to biological selection.

Creative destruction also brings three philosophical considerations to the table: (1) technology is complex and unpredictable, (2) technology is potential dangerous to itself and (3) the nature of technology is classist, harming the poor in some instances.
We consider the third point to be most important and most easily observed.
Technology has the power to do great good and great harm. 
The good and harm can stem from individual technologies themselves, like a technology that cures cancer is great, but drones that kill humans are arguably bad. 
Zooming out, the nature of technology and how it evolves has a devestating effect on the poor.
The poor cannot afford the new technologies, and the new technologies often cause firms to not produce old technologies because they are not economically viable, leaving the poor without anything. 
Here, it is the process by which technology is created and evolves which harms the poor, not the technology itself.

\bibliographystyle{apalike}
\bibliography{paper}

\end{document}















